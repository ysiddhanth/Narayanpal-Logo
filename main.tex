\documentclass[conference]{IEEEtran}
\IEEEoverridecommandlockouts

% --- Packages (consistent with first file) ---
\usepackage{setspace}
\usepackage{gensymb}
\singlespacing
\usepackage[cmex10]{amsmath}
\usepackage{amsthm}
\usepackage{mathrsfs}
\usepackage{txfonts}
\usepackage{stfloats}
\usepackage{bm}
\usepackage{cite}
\usepackage{cases}
\usepackage{subfig}
\usepackage{longtable}
\usepackage{multirow}
\usepackage{enumitem}
\usepackage{mathtools}
\usepackage{tikz}
\usepackage{circuitikz}
\usepackage{verbatim}
\usepackage[breaklinks=true]{hyperref}
\usepackage{tkz-euclide}
\usepackage{listings}
\usepackage{color}
\usepackage{array}
\usepackage{calc}
\usepackage{hhline}
\usepackage{ifthen}
\usepackage{lscape}
\usepackage{chngcntr}
\usepackage{algorithm}
\usepackage[indLines=false]{algpseudocodex}

% --- Custom Math Commands (from first file) ---
\DeclareMathOperator*{\Res}{Res}
\renewcommand\thesection{\arabic{section}}
\renewcommand\thesubsection{\thesection.\arabic{subsection}}
\renewcommand\thesubsubsection{\thesubsection.\arabic{subsubsection}}
\renewcommand\thetable{\arabic{table}}
\hyphenation{op-tical net-works semi-conduc-tor}
\def\inputGnumericTable{}

\newtheorem{theorem}{Theorem}[section]
\newtheorem{problem}{Problem}
\newtheorem{proposition}{Proposition}[section]
\newtheorem{lemma}{Lemma}[section]
\newtheorem{corollary}[theorem]{Corollary}
\newtheorem{example}{Example}[section]
\newtheorem{definition}[problem]{Definition}
\theoremstyle{remark}
\newtheorem{rem}{Remark}

\newcommand{\BEQA}{\begin{eqnarray}}
	\newcommand{\EEQA}{\end{eqnarray}}
\providecommand{\mbf}{\mathbf}
\providecommand{\pr}[1]{\ensuremath{\Pr\left(#1\right)}}
\providecommand{\qfunc}[1]{\ensuremath{Q\left(#1\right)}}
\providecommand{\sbrak}[1]{\ensuremath{{}\left[#1\right]}}
\providecommand{\lsbrak}[1]{\ensuremath{{}\left[#1\right.}}
\providecommand{\rsbrak}[1]{\ensuremath{{}\left.#1\right]}}
\providecommand{\brak}[1]{\ensuremath{\left(#1\right)}}
\providecommand{\lbrak}[1]{\ensuremath{\left(#1\right.}}
\providecommand{\rbrak}[1]{\ensuremath{\left.#1\right)}}
\providecommand{\cbrak}[1]{\ensuremath{\left\{#1\right\}}}
\providecommand{\lcbrak}[1]{\ensuremath{\left\{#1\right.}}
\providecommand{\rcbrak}[1]{\ensuremath{\left.#1\right\}}}
\newcommand{\sgn}{\mathop{\mathrm{sgn}}}
\providecommand{\abs}[1]{\left\vert#1\right\vert}
\providecommand{\res}[1]{\Res\displaylimits_{#1}} 
\providecommand{\norm}[1]{\left\lVert#1\right\rVert}
\providecommand{\mtx}[1]{\mathbf{#1}}
\providecommand{\mean}[1]{E\left[ #1 \right]}   
\providecommand{\fourier}{\overset{\mathcal{F}}{ \rightleftharpoons}}
\providecommand{\system}[1]{\overset{\mathcal{#1}}{ \longleftrightarrow}}
\newcommand{\solution}{\noindent \textbf{Solution: }}
\newcommand{\cosec}{\,\text{cosec}\,}
\providecommand{\dec}[2]{\ensuremath{\overset{#1}{\underset{#2}{\gtrless}}}}
\newcommand{\myvec}[1]{\ensuremath{\begin{pmatrix}#1\end{pmatrix}}}
\newcommand{\mydet}[1]{\ensuremath{\begin{vmatrix}#1\end{vmatrix}}}
\renewcommand{\vec}[1]{\boldsymbol{\mathbf{#1}}}

% --- Document Begins ---
\begin{document}
	
	\title{A Logo for Narayanpal}
	\author{
		\IEEEauthorblockN{G.V.V. Sharma} 
			\IEEEauthorblockA{Department of Electrical Engineering, 
				\\
				Indian Institute of Technology Hyderabad,\\
				Kandi, India 502284
				\\
				gadepall@ee.iith.ac.in}
		}
	\maketitle
	
	\begin{abstract}
		This paper determines the parameter pairs $(a, b)$ for the function $f(t) = e^{-at}u(t) + e^{bt}u(-t)$, when subject to normalization, such that these pairs correspond to the endpoints of the latus recta of an associated conic. The work derives the conic equation binding the parameters, applies eigen-decomposition, and uses affine transformations to identify the valid $(a, b)$ values.
	\end{abstract}

	\section{Question}
	Given that
	\begin{align}
		f(t) &= e^{-at}u(t) + e^{bt}u(-t) \label{q1}\\
		u(t) &= \begin{cases}
			0, & t<0\\
			\frac{1}{2}, & t=0 \\
			1, & t>0
		\end{cases} \label{q2}\\
		\int_{-\infty}^{\infty} f(t) &= 1 \label{q3}
	\end{align}
	
	Find the possible values of $(a,b)$ if these are the end points of the latus recta of the associated conic. Plot $f(t)$ for these values of $(a,b)$.
	
	\section{Solution}
	We expand the integral as 
	\begin{align}
		\int_{-\infty}^{\infty} f(t) &= \int_{-\infty}^{0} f(t) + \int_{0}^{\infty} f(t)\\
		&= \int_{-\infty}^{0} e^{bt} + \int_{0}^{\infty} e^{-at}\\
		&= \frac{1}{b} + \frac{1}{a} \label{integral_result}
	\end{align}
	Substituting \eqref{q3} in \eqref{integral_result}:
	\begin{align}
		\frac{1}{a} + \frac{1}{b} &= 1\\
		ab - a - b &= 0 \label{conic_ab}
	\end{align}
	This is the equation of a conic. If we take $a$ as $x$ and $b$ as $y$ and express this as a conic in standard form, we get
	\begin{align}
		\text{g}(\vec{x}) = \vec{x}^\text{T}\vec{Vx} + 2\vec{u}^\text{T}\vec{x} + f \label{actual_conic}
	\end{align}
	By comparison:
	\begin{align}
		\vec{V} &= \myvec{0 & \frac{1}{2} \\ \frac{1}{2} & 0}\\
		\vec{u} &= \myvec{-\frac{1}{2} \\ -\frac{1}{2}}\\
		f &= 0
	\end{align}
	We eigen-decompose $\vec{V}$ as 
	\begin{align}
		\vec{V} &= \vec{PDP}^\text{T}\\
		\vec{P} &= \myvec{\frac{1}{\sqrt{2}} & \frac{-1}{\sqrt{2}} \\ \frac{1}{\sqrt{2}} & \frac{1}{\sqrt{2}}}\\
		\vec{D} &= \myvec{\frac{1}{2} & 0 \\ 0 & \frac{-1}{2}}
	\end{align}
	Convert the conic into a standard conic using affine transformations.
	\begin{align}
		\vec{y}^\text{T}\brak{\frac{\vec{D}}{f_0}}\vec{y} &= 1\\
		\vec{x} &= \vec{Py} + \vec{c} \label{t1}
	\end{align}
	Where 
	\begin{align}
		f_0 &= \vec{u}^\text{T}\vec{V}^{-1}\vec{u} - f = 1\\
		\vec{c} &= -\vec{V}^{-1}\vec{u} = \myvec{1 \\ 1}
	\end{align}
	The eigenvalues of $\vec{D}$ are $\lambda_1 = \frac{1}{2}$, $\lambda_2=-\frac{1}{2}$. Using a reflection matrix and further transformation, we get the hyperbola in standard form:
	\begin{align}
		\vec{z}^\text{T}\brak{\frac{\vec{D_0}}{f_0}}\vec{z} &= 1\\
		j(\vec{z}) = \vec{z}^\text{T}\vec{D_0}\vec{z} - f_0 &= 0  \label{standard_conic}\\
		\vec{y} &= \vec{P}_0\vec{z} \label{t2}
	\end{align}
	Here $\vec{P}_0 = \myvec{0 & 1 \\ 1 & 0}$ and $\vec{D}_0 = \myvec{\frac{-1}{2} & 0 \\ 0 & \frac{1}{2}}$.
	
	Now, solve for the endpoints of the latus recta:
	\begin{align}
		\vec{n} &= \sqrt{\lambda_2}\vec{p}_1 = \myvec{0 \\ \frac{1}{\sqrt{2}}}\\
		e &= \sqrt{2}\\
		c &= \pm \frac{1}{\sqrt{2}}\\
		\vec{F} &= \pm 2 \vec{e}_2
	\end{align}
	Equation of latus recta:
	\begin{align}
		\vec{n}^\text{T}\vec{x} &= \vec{n}^\text{T}\vec{F}\\
		\equiv \vec{x} &= \vec{h} + k\vec{m} \label{latus_rectum}\\
		\vec{h} &= \myvec{0 \\ \pm 2}\\
		\vec{m} &= \myvec{1 \\ 0}
	\end{align}
	Let $\vec{\hat{z}}$ be the endpoints of the latus recta:
	\begin{align}
		k &= \pm \sqrt{2}\\
		\therefore \vec{\hat{z}} &= \myvec{\pm \sqrt{2} \\ \pm 2}
	\end{align}
	Transforming back to the original conic:
	\begin{align}
		\vec{\hat{x}} &= \vec{P}\brak{\vec{P}_0 \vec{\hat{z}}} + \vec{c}
	\end{align}
	Which gives:
	\begin{align*}
		\vec{\hat{x}}_1 &= \myvec{2+\sqrt{2} \\ \sqrt{2}}\\
		\vec{\hat{x}}_2 &= \myvec{\sqrt{2} \\ 2+\sqrt{2}}\\
		\vec{\hat{x}}_3 &= \myvec{2-\sqrt{2} \\ -\sqrt{2}}\\
		\vec{\hat{x}}_4 &= \myvec{-\sqrt{2} \\ 2-\sqrt{2}}
	\end{align*}
	Only $\vec{\hat{x}}_1$ and $\vec{\hat{x}}_2$ are valid as negative $a$ or $b$ will not yield a finite $f(t)$.
	
	\section{Plots}
	\begin{figure}[H]
		\centering
		\includegraphics[width=1.2\columnwidth]{figs/Hyperbola.png} 
		\caption{Conic Section}
		\label{fig:Hyperbola}
	\end{figure}
	
	\begin{figure}[H]
		\centering
		\includegraphics[width=1\columnwidth]{figs/function.png} 
		\caption{Function $f(t)$ for valid $(a, b)$}
		\label{fig:Function}
	\end{figure}
	
	% Bibliography style
	\bibliographystyle{IEEEtran}
	
\end{document}
